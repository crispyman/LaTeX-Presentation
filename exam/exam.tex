\documentclass{exam}

\usepackage{amsmath}
 
\begin{document}
 
\begin{center}
\fbox{\fbox{\parbox{5.5in}{\centering
Answer the questions in the spaces provided. If you run out of room
for an answer, continue on the back of the page.}}}
\end{center}
% vspace creates a vertical space between two lines 
\vspace{5mm}
% enspace ensures there is a space after the colon
% and before line generated by hrulefill which goes
% all the way to the end of the page
% A known "FEATURE" with comments is that the next line
% after a comment is not indendted so have whitespace
% after comments and before text

Name and section:\enspace\hrulefill
 
\vspace{5mm}
 
Instructor’s name:\enspace\hrulefill

% tells latex to start keeping track of points
% otherwise grade table won't work
\addpoints
 
\begin{questions}
\question[20] Which of these guys published a paper on Browninan Motion

\begin{checkboxes}
 \choice Stephen Hawking 
 \choice Albert Einstein
 \choice Isaac Newton
 \choice I don't know
\end{checkboxes}

\question[10] Is it true that \(x^n + y^n = z^n\) if \(x,y,z\) and \(n\) are
positive integers?. Explain.

% each call of stretch one divides 
% up the space available on the page
% equaly if you used 2 for one section
% then its space will be equal to two
% calls to stretch 1.
\vspace{\stretch{1}}
 
\question[10] Prove that the real part of all non-trivial zeros of the function
\(\zeta(z)\) is \(\frac{1}{2}\)

\vspace{\stretch{1}} 

\question[10] Prove $\begin{pmatrix} -24 & 35 & 35 \end{pmatrix}\begin{pmatrix}-5&-6&-2\\3&3&1\\-6&-7&-2\end{pmatrix} = \begin{pmatrix}15 & 4 & 13 \end{pmatrix}$

\vspace{\stretch{2}}

\bonusquestion[10] what is x when y equals 5: $y = \frac{1}{2} * x$

\vspace{\stretch{1}}


\end{questions}

\begin{center}
\combinedgradetable[h][questions]
\end{center}

\end{document}
